%% Copyright (C) 2010-2012, Gostai S.A.S.
%%
%% This software is provided "as is" without warranty of any kind,
%% either expressed or implied, including but not limited to the
%% implied warranties of fitness for a particular purpose.
%%
%% See the LICENSE file for more information.

\chapter{Getting Started}
\label{sec:tut:started}

\us comes with a set of tools, two of which being of particular
importance:
\begin{description}
\item[\dfn{urbi}] launches an \urbi server.  There are several means
  to interact with it, which we will see later.
\item[\dfn{urbi-launch}] runs \urbi components, the UObjects, and connects
  them to an \urbi server.
\end{description}

Please, first make sure that these tools are properly installed.  If you
encounter problems, please see the frequently asked questions
(\autoref{sec:faq}), and the detailed installation instructions
(\autoref{sec:installation}).

\begin{shell}
# Make sure urbi is properly installed.
$ urbi --version
Urbi version 3.x.y 
\end{shell}%$

There are several means to interact with a server spawned by \command{urbi},
see \autoref{sec:tools:urbi} for details.  First of all, you may use the
options \option{-e}/\option{--expression \var{code}} and
\option{-f}/\option{--file \var{file}} to send some \var{code} or the
contents of some \var{file} to the newly run server.  The option
\option{q}/\option{--quiet} discards the banner.

You may combine any number of these options, but beware that being
event-driven, the server does not ``know'' when a program ends.  Therefore,
batch programs should end by calling \lstinline{shutdown}.  Using a Unix
shell:

\begin{shell}[alsolanguage={[interactive]urbiscript},caption={A batch session under Unix.}]
# A classical program.
$ urbi -q -e 'echo("Hello, World!");' -e 'shutdown;'
[00000004] *** Hello, World!
\end{shell}

If you are running Windows, then, since the quotation rules differ, run:

\begin{shell}[alsolanguage={[interactive]urbiscript},caption={A batch session under Windows.}]
# A classical program.
$ urbi -q -e "echo(""Hello, World!"");" -e "shutdown;"
[00000004] *** Hello, World!
\end{shell}


To run an interactive session, use option
\option{-i}/\option{--interactive}.  Like most interactive interpreters,
\urbi will evaluate the given commands and print out the results.

\begin{shell}[alsolanguage={[interactive]urbiscript},caption={An interactive session under Unix.}]
$ urbi -i
[00000825] *** Urbi version 3.x.y
1+2;
[00001200] 3
shutdown;
\end{shell}%$

The output from the server is prefixed by a number surrounded by square
brackets: this is the date (in milliseconds since the server was launched)
at which that line was sent by the server.  This is useful at occasions,
since \urbi is meant to run many parallel commands.  Since these timestamps
are irrelevant in most examples, they will often be filled with zeroes
through this documentation.

Under Unix, the program \command{rlwrap} provides additional services
(history of commands, advanced command line edition etc.); run \samp{rlwrap
  urbi -i}.

In either case the server can also be made available for network-based
interactions using option \option{--port \var{port}}.  Note that while
\lstinline{shutdown} asks the server to quit, \lstinline{quit} only quits
one interactive session.  In the following example (under Unix) the server
is still available for other, possibly concurrent, sessions.

\begin{shell}[alsolanguage={[interactive]urbiscript},caption={An interactive session under Unix.}]
$ urbi --port 54000 &
[1] 77024
$ telnet localhost 54000
Trying 127.0.0.1...
Connected to localhost.
Escape character is '^]'.
[00004816] *** Urbi version 3.x.y
12345679*8;
[00018032] 98765432
quit;
Connection closed by foreign host.
\end{shell}%$

Under Windows, instead of using \command{telnet}, you may use
\command{Gostai Console} (part of the package), which provides a Graphical
User Interface to a network-connection to an \urbi server.  To launch the
server, run:

\begin{shell}[alsolanguage={[interactive]urbiscript},caption={Starting an interactive session under Windows.}]
C:\...> start urbi --port 54000
\end{shell}

\noindent
and to launch the client, click on \command{Gostai Console} which is
installed by the installer.

Then, the interaction proceeds in the \command{Gostai Console} windows.
Specify the host name and port to use (\samp{127.0.0.1:54000}) in the text
field in the top of the window and click on the right to start the
connection.

\begin{center}
  \includegraphics[width=.8\linewidth]{img/gostai-console}
\end{center}

The program \command{urbi-send} (see \autoref{sec:tools:urbi-send}) provides
a nice interface to send batches of instructions (and/or files) to a running
server.

\begin{shell}[alsolanguage={[interactive]urbiscript}]
$ urbi-send -P 54000 -e "1+2*3;" -Q
[00018032] 7
# Have the server shutdown;
$ urbi-send -P 54000 -e "shutdown;"
\end{shell}

\medskip

You can now send commands to your \urbi server. If at any point you get
lost, or want a fresh start, you can simply close and reopen your connection
to the server to get a clean environment.  In some cases, particularly if
you made global changes in the environment, it is simpler to start anew:
shut your current server down using the command \lstinline{shutdown}, and
spawn a new one. In interactive mode you can also use the shortcut sequence
\key{Ctrl-D}, like in many other interpreters.

In case of a foreground task preventing you to execute other commands, you
can use \key{Ctrl-C} to kill the foreground task, clear queued commands and
restore interactive mode.

\medskip

You are now ready to proceed to the \us tutorial: see \autoref{part:tut}.

Enjoy!

%%% Local Variables:
%%% coding: utf-8
%%% mode: latex
%%% TeX-master: "urbi-sdk"
%%% ispell-dictionary: "american"
%%% ispell-personal-dictionary: "urbi.dict"
%%% fill-column: 76
%%% End:
